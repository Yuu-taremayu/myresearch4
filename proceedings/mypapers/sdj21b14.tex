%% Created by Yasuyuki SAITO, Department of Information Engineering.


%% 2016/01/13 Edit by Kouta ASAI, and Akira NEMOTO, Advanced Control and Information Engineering Course
%% 2017/12/07 Edit by by Shinichi OEDA, Department of Information Engineering.
%% 「★」マークが変更を加えた部分を表す

%% ★jsarticleに変更 fleqnで数式を左寄せにする
\documentclass[twocolumn, fleqn, uplatex]{jsarticle}

\usepackage{proceeding-DJ}
\usepackage{afterpage}
%% ★数式を使う人が多いと思うので
\usepackage{amsmath}
\usepackage{bm}
\usepackage{booktabs}
\usepackage{enumitem}
\usepackage{subcaption}


%% ★ヒラギノを使っている人向け
\usepackage[deluxe, expert]{otf}

% \makeatletter
% \long\def\@makecaption#1#2{%
%   \vskip\abovecaptionskip
%   \iftdir\sbox\@tempboxa{#1\hskip1zw#2}%
%     \else\sbox\@tempboxa{#1~ #2}%
%   \fi
%   \ifdim \wd\@tempboxa >\hsize
%     \iftdir #1\hskip1zw#2\relax\par
%     \else #1~ #2\relax\par\fi
%   \else
%   \global \@minipagefalse
%   \hbox to\hsize{\hfil\box\@tempboxa\hfil}%
% \fi
% \vskip\belowcaptionskip}
% \makeatother

\begin{document}

\pagestyle{empty}

%% 番号をつけるときはこのすぐ下の行を有効にして,番号を編集する.
%%\date{J--33}
%%所属をDJ以外にしたい場合は,styファイルを編集する.
\date{}
\titleJ{秘密分散法におけるユーザフレンドリーなUIの提案・評価・考察}
\titleE{}
\authorJ{三浦 夢生}
\authorE{Yu Miura}
\abstract{The purpose of this study is ... \\
%
\hspace*{1zw} If you would like to start a new paragraph, you should use
``\texttt{\textbackslash hspace*\{1zw\}}''.}
% \keywords{The key word1 of the test for \LaTeXe \ template file, The key
% word2 of the test for \LaTeXe \ template file, The key word3 of the test
% for \LaTeXe \ template file}
\keywords{Secret Sharing, cryptgraphy, Shamir's Secret Sharing, User Interface}
\maketitle

%% 現在ページの上部へのフロートの配置を抑制.
%% ここに記述しておくことで,最初のページの左段上部に図表を置かない.
\suppressfloats[t]

\section{まえがき}
\section{前提知識}
\subsection{Shamirの秘密分散法}
\subsection{拡大体}
\section{実装}
\section{評価方法}
\section{結果・考察}
\section{まとめ}
\section{今後の展望}

\begin{thebibliography}{9}
\end{thebibliography}

\end{document}
