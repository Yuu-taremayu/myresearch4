%% Created by Yasuyuki SAITO, Department of Information Engineering.


%% 2016/01/13 Edit by Kouta ASAI, and Akira NEMOTO, Advanced Control and Information Engineering Course
%% 2017/12/07 Edit by by Shinichi OEDA, Department of Information Engineering.
%% 「★」マークが変更を加えた部分を表す

%% ★jsarticleに変更 fleqnで数式を左寄せにする
\documentclass[twocolumn, fleqn, uplatex]{jsarticle}

\usepackage{proceeding-DJ}
\usepackage{afterpage}
%% ★数式を使う人が多いと思うので
\usepackage{amsmath}
\usepackage{bm}
\usepackage{booktabs}
\usepackage{enumitem}
\usepackage{subcaption}


%% ★ヒラギノを使っている人向け
\usepackage[deluxe, expert]{otf}

% \makeatletter
% \long\def\@makecaption#1#2{%
%   \vskip\abovecaptionskip
%   \iftdir\sbox\@tempboxa{#1\hskip1zw#2}%
%     \else\sbox\@tempboxa{#1~ #2}%
%   \fi
%   \ifdim \wd\@tempboxa >\hsize
%     \iftdir #1\hskip1zw#2\relax\par
%     \else #1~ #2\relax\par\fi
%   \else
%   \global \@minipagefalse
%   \hbox to\hsize{\hfil\box\@tempboxa\hfil}%
% \fi
% \vskip\belowcaptionskip}
% \makeatother

\begin{document}

\pagestyle{empty}

%% 番号をつけるときはこのすぐ下の行を有効にして,番号を編集する.
%%\date{J--33}
%%所属をDJ以外にしたい場合は,styファイルを編集する.
\date{}
\titleJ{秘密分散法構築におけるユーザフレンドリーなUIの提案・評価・考察}
\titleE{Proposal, Evaluation and Discussion of user-friendly UI for Secret Sharing scheme construction}
\authorJ{三浦 夢生}
\authorE{Yu Miura}
\abstract{In modern society, spreads cloud or online service and a lot of important information (e.g. personal data) is transmitted and received.
Cyber-attacks against companies and organizations are increasing year by year, and the damage caused by ransomware, one of them, is expanding.
In this study, I deals with a scheme called secret sharing.
By using this scheme, we can reduce the risk of loss or theft of the secret. and allows managers to adjust discretion per role by using hierarchical one.
The purpose is designing intuitive and user-friendly UI for people who don't know too much about secret sharing and proposing easy to use secret sharing tool even individuals.
In an evaluation, I ask testers to use the proposed UI and the conventional CLI-based UI, and conduct a questionnaire survey and analysis of whether the UI is "visually easy to understand", "easy to operate", and "time required to build a secret sharing system".
}
% \keywords{The key word1 of the test for \LaTeXe \ template file, The key
% word2 of the test for \LaTeXe \ template file, The key word3 of the test
% for \LaTeXe \ template file}
\keywords{Secret Sharing, cryptgraphy, Shamir's Secret Sharing, User Interface}
\maketitle

%% 現在ページの上部へのフロートの配置を抑制.
%% ここに記述しておくことで,最初のページの左段上部に図表を置かない.
\suppressfloats[t]

\section{まえがき}
近年,クラウド及びオンラインサービスが%
企業や一般家庭において広く普及し\cite{lit:soumu-1},\cite{lit:soumu-2},%
個人情報などの多くの重要な情報がやり取りされている.%
その中で企業・団体等に対するサイバー攻撃は年々増加しており,%
その中の一つであるランサムウェアによる被害は拡大している\cite{lit:keisatu}.%
IPAの情報セキュリティ10大脅威においても,機密情報を狙った攻撃は%
大きな影響を与えるとされる\cite{lit:ipasec-1}.%
これらは機密情報やアクセス権の集中管理によって攻撃を受けた際のリスクが%
高まるためであり,最小限の適切なアクセス管理が重要である.%
しかし,無闇な機密情報のコピーや,アクセス権限の厳重化では%
かえってリスクが高まる\cite{lit:ipasec-2}.%

これらのリスク低減のため機密情報の分散管理及びアクセス権限の柔軟な付与が%
できる「秘密分散\cite{lit:nishikawa}」という技術を用いる.%
秘密分散によって情報の漏洩・紛失のリスクを低減し,%
またこれを階層化することによって管理者の役割ごとの裁量を調節できる手法を%
UIのバックエンドに採用する.%
フロントエンドサイドでは,秘密分散を詳しく知らないユーザでも%
直感的に利用しやすいUIの設計をする.%
本研究では個人で気軽に利用できるような秘密分散ツールの設計・提案を行い,%
UIによってどの程度秘密分散ツールが利用しやすくなったかアンケート調査し,%
提案UIの有用性や改善点について考察することを目的とする.

実装には,ユーザの環境を選ばずブラウザで動作するJavascriptと,%
そのフレームワークであるReact.jsを主として用いる.%
React.jsはGUI作成に特化したJavascriptフレームワークであり,%
SPA(Single Page Application)を作成しやすい画面遷移のない実装が可能となり,%
ユーザ側での画面読み込みや通信負荷を減らすことができる.%

アンケート調査ではコマンドラインライクなUIと木構造を模した提案UIを用意し,
それぞれファイルを秘密分散する課題を遂行してもらい,%
それにかかった時間の計測,使用感に関する質問への回答を求める.

\section{前提知識}
\subsection{Shamirの秘密分散法}
秘密分散法とは,まず対象となる秘密情報をあるアルゴリズムを用いて%
シェアと呼ばれる分散情報を作成し,%
そのシェアを管理する人・端末に配布し管理・保管する手法である.%
元の秘密情報を得る際には,管理されたシェアを必要な数だけ集め,%
復元アルゴリズムによって復元する.

いくつかある秘密分散法のうち,本研究ではShamirの秘密分散法\cite{lit:shamir}を用いている.%
この手法は$(k,n)$しきい値秘密分散法とも呼ばれ,シェアを$n$個生成し,%
$k$個のシェアを集めることで秘密情報の復元が可能であるが,%
$(k-1)$個以下のシェアからは復元は不可能である.

シェアを生成するアルゴリズムは,まずEq.\ref{randompoly}のような%
秘密情報$S$を定数項とするランダムな$k-1$次多項式を定める.

\begin{equation}
f(x) = S + a_{1}x + a_{2}x^{2} + \dots + a_{k-1}x^{k-1} \label{randompoly}
\end{equation}

管理に参加する人や端末の数$i$に対し,%
$f(1),f(2),\dots,f(i)$を計算し,配布する.

復元の際は,$k$個のシェア$(j,f(j))$を集めて$k$個の式を立て,%
連立方程式を解くことで秘密情報$S$を得る.
$k$個のシェアは同じ多項式から生成されたものであればよく,%
任意のものを用いればよい.

またこのときラグランジュ補間を用い,Eq.\ref{lagrange-1},Eq.\ref{lagrange-2}%
において$x=0$の場合を計算することで,多項式のうち定数項のみ,%
つまり元の秘密情報を得ることができる.

\begin{equation}
L(x) = \sum_{i=0}^{k}y_{i}l_{i}(x) \label{lagrange-1}
\end{equation}

\begin{equation}
l_{i}(x) = \prod_{j=0,j{\neq}i}^{k}{\frac{x-x_{j}}{x_{i}-x_{j}}} \label{lagrange-2}
\end{equation}

\subsection{拡大体}
体とは集合に対して加法,乗法の二つの二項演算を定めた%
代数的構造のことであり,加法,乗法のどちらに関しても%
結合法則・交換法則・分配法則が成り立ち,単位元及び逆元が存在する.%
また前提として,集合の二つの要素に対して,その二項演算の結果が%
集合の中に存在することも条件である.%
これらのことから,体の要素は加減乗除に閉じているといえる\cite{nozaki-1}.%
これを踏まえて拡大体$K'$とは,ある体$K$に代数的構造を損なわないように%
元${\alpha}{\notin}K$を追加して得られる体のことである.%
このとき,$K'$は$K$を含んでおり,$K$は$K'$の部分体ともいう.%
また,$K'$では$K$で定義された演算をそのまま用いることができる\cite{lit:siozaki}.

本研究では$GF(2)=\{0,1\}$という二つの要素をもつ体を拡大し%
$GF(2^8)$という集合を扱う.まず$GF(2)$における演算を%
Table.\ref{tab:addition}及びTable.\ref{tab:multiplication}のように定義する.%
この体は整数を$2$で割った剰余によって構成された集合と見ることもできる.

\begin{table}[htbp]
	\begin{minipage}[t]{0.45\hsize}
		\caption{加法の演算表}
		\centering
		\scalebox{0.8}{
		\begin{tabular}{|c|c|c|}
			\hline
			+ & 0 & 1 \\
			\hline
			0 & 0 & 1 \\
			\hline
			1 & 1 & 0 \\
			\hline
		\end{tabular}}
		\label{tab:addition}
	\end{minipage}
	%
	\hfill
	%
	\begin{minipage}[t]{0.45\hsize}
		\caption{乗法の演算表}
		\centering
		\scalebox{0.8}{
		\begin{tabular}{|c|c|c|}
			\hline
			+ & 0 & 1 \\
			\hline
			0 & 0 & 1 \\
			\hline
			1 & 1 & 0 \\
			\hline
		\end{tabular}}
		\label{tab:multiplication}
	\end{minipage}
\end{table}

この演算を踏まえて$GF(2)$を拡大し$GF(2^8)$を構成する.%
このとき,既約多項式という概念を導入すると,$GF(2^8)$において%
$2$の剰余の集合と扱えたように,$GF(2^8)$は既約多項式による剰余の%
集合とすることができる.本研究では既約多項式%
$x^{8}+x^{4}+x^{3}+x^{2}+1$を用いたため,$GF(2^8)$の各元は%
係数が$0$または$1$の$7$次式である.

$GF(2^8)$において,加法はTable.\ref{tab:addition}より各次数の項の%
排他的論理和となるため,$7$次までの項の係数を8bitの列として見ることで,%
コンピュータで扱いやすい演算となる.減法について,$1{\equiv}−1\mod2$%
であるため,加法と同じ演算を用いることができる.

乗法は各元の算術的な積をとり,%
既約多項式による剰余を求めることで定義される.%
除法についても$0$除算を除いて体の定義から逆元が存在するため%
剰余の計算をすることで求めることができる.

ここで仮想的に既約多項式の根$\alpha$を考える.%
この根は体上には存在しないが,この根は体の原始元であり,%
体の要素は原始元のべき乗によって巡回的に生成される.%
このことから,多項式による表記と原始元のべき乗による表記を%
対応させることで多項式の積を実装せずに体上の乗除が可能となる\cite{lit:nozaki-2}.

\section{実装}
\subsection{バックエンド}
\subsection{フロントエンド}

\section{アンケート調査}

\section{結果・考察}
実験途中により結果は省略
\section{まとめ}
\section{今後の展望}

\begin{thebibliography}{9}
	\footnotesize
	\bibitem{lit:soumu-1}
		総務省, 
		``デジタルで支える暮らしと経済'', 
		情報通信白書, pp.156-159, July 2020.

	\bibitem{lit:soumu-2}
		総務省, 
		``デジタルで支える暮らしと経済'', 
		情報通信白書, pp.313-315, July 2020.
	
	\bibitem{lit:keisatu}
		警察庁, 
		``令和3年におけるサイバー空間をめぐる脅威の情勢等について'', 
		警察庁 広報資料, pp.3-5, April 2022.
	
	\bibitem{lit:ipasec-1}
		独立行政法人情報処理推進機構セキュリティセンター, 
		``情報セキュリティ10大脅威'', 
		情報処理推進機構, pp.36-39, March 2022.

	\bibitem{lit:ipasec-2}
		独立行政法人情報処理推進機構, 
		``組織における内部不正防止ガイドライン'', 
		情報処理推進機構, pp.40-44, April 2022.
	
	\bibitem{lit:nishikawa}
		西川律子,
		``秘密分散法の概要'', 
		沖テクニカルレビュー 第205号, pp.70-71, January 2006.

	\bibitem{lit:ipa}
		独立行政法人情報処理推進機構, 
		``企業における営業秘密に関する実態調査2020'', 
		調査実態報告書, pp.44-47, March 2020.
	
	\bibitem{lit:maruhi}
		株式会社エスロジカル, 
		``マル秘分散'', 
		http://www.ma-bu.com/, 
		閲覧日Dec 2021.
	
	\bibitem{lit:ssss}
		Jon Frisby, 
		``ssss'', 
		https://github.com/MrJoy/ssss, 
		閲覧日Dec 2021.
	
	\bibitem{lit:shamir}
		A Shamir, 
		``How to share a secret'', 
		Communications of ACM, Vol.22, pp.612-613, April 1979.
	
	\bibitem{lit:nozaki-1}
		野崎昭宏, 
		``納得する群・環・体'', 
		講談社, 第1版, pp.136-137, May 2016.
	
	\bibitem{lit:siozaki}
		汐崎陽, 
		``情報・符号理論の基礎'', 
		オーム社, 第2版, pp.90-91, May 2019.
	
	\bibitem{lit:nozaki-2}
		野崎昭宏, 
		``納得する群・環・体'', 
		講談社, 第1版, pp.160-176, May 2016.

	\bibitem{lit:takaara}
		高荒亮, 岩村惠一, 
		``XORを用いた高速な$(k,L,n)$ランプ型秘密分散法に関する研究'', 
		コンピュータセキュリティシンポジウム2009(CS2009)論文集, Vol.2009, pp.1-6, October 2011.
	
	\bibitem{lit:aono}
		青野成俊, 岩村惠一, 
		``実用観点からみた秘密分散法に関する一考察'', 
		コンピュータセキュリティシンポジウム2009(CS2009)論文集, Vol.2009, pp.1-6, October 2011.

\end{thebibliography}

\end{document}
